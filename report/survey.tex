\documentclass[11pt]{article}
\usepackage[T1]{fontenc}
\usepackage{lmodern}
\usepackage[margin=1in, headheight=15pt]{geometry}
\usepackage{amsmath, amsthm, amsfonts}
\usepackage{hyperref}
\usepackage{pgfplotstable}
\pgfplotsset{compat=1.12}
\usepackage{booktabs}
\usepackage{footnote}

\usepackage{fancyhdr, setspace}
\renewcommand{\headrulewidth}{0pt}
\pagestyle{fancy}
\setlength{\headheight}{15pt} % Removes warnings. Set to 30pt if 2 lines on the header.
\lhead{}
\rhead{}
\usepackage{totcount}
\regtotcounter{page}
\cfoot{\ifnum\totvalue{page} > 1 \thepage \else\fi}

\usepackage{enumitem}
\setlist{parsep=0pt, listparindent=0.5cm}

\begin{document}
\begin{center}
{ \huge
	Simplifying Parallel Graph Processing: \\
}
{ \Large
	Survey of Existing Platforms \\
}
Sam Pollard (\href{mailto:spollard@cs.uoregon.edu}{spollard@cs.uoregon.edu}), University of Oregon \\
\today
\end{center}

This is a survey of existing graph analytics frameworks.

\section{Machine Specifications}
Below are the preliminary results by running two benchmarks on the research computer Arya.

% TODO: Make this a pgfplotstable with an automatically generated csv.
\begin{table}[htb]
	\centering
% Arya
%	\begin{tabular}{l|r}
%		CPU & 72 Core Intel Xeon E5 2699 v3\\ \hline
%		RAM & 256 GB DDR4 2133 MHz
%	\end{tabular}

% Keep in mind you can do this at the beginning: string replace={s1}{s2}
	\pgfplotstabletypeset[
	header=false,
	col sep=tab,
	string type,
    every head row/.style={output empty row, before row=\bottomrule},
	columns/0/.style={column type={|r|}},
	columns/1/.style={column type={l|}},
	every last row/.style={after row=\toprule},
	]{specs.csv}
	\caption{Machine specifications.}
	\label{tab:specs}
\end{table}

\begin{table}[htb]
	\centering
	\begin{tabular}{lr}
		Transport & \\ \hline
		Network Topology & \\ \hline
		Local Scheduling & \\ \hline
		Runtime Feedback & \\ \hline
		Approach & \\ \hline
		Algorithmic Considerations & \\ \hline
	\end{tabular}
	\caption{Middleware specifications}
	\label{tab:reportcard}
\end{table}

Performance in millions of traversed edges per second (MTEPS)

% TODO: This should be unnecessary once autogeneration is used.
\begin{table}[htb]
	\centering
	\begin{tabular}{l|r|r|}
	 & PowerGraph & OpenG \\ \hline
	BFS & $87.4$ & $341$ \\ \hline
	SSSP & $1.09$ & $3.08$ \\ \hline
	\end{tabular}
	\caption{Performance Results}
	\label{tab:perf}
\end{table}

\section{Graph Processing Taxonomy}
This is in the spirit of \cite{Doekemeijer:2015:GPFSurvey}. Here, ``|'' means ``or'' and ``+'' means ``and.'' FOSS means Free and Open Source Software. The quotes around ``yes'' for HPC mean that the product claims to be amenable to high performance computing. Whether these actually achieve their goal is one of the purposes of this project.
\begin{savenotes}
	\begin{table}
		\centering
		% \small
		\pgfplotstabletypeset[
			col sep=comma,
			string type, % Makes the .style={string type} redundant
			every head row/.style={after row=\midrule},
			columns/Name/.style={string type, column type={l|}},
			columns/Type/.style={string type},
			columns/HPC/.style={string type},
			columns/Parallelism/.style={string type},
			columns/Target/.style={string type},
			columns/FOSS/.style={string type},
			columns/Source/.style={string type},
			columns/Notes/.style={
				preproc cell content/.style={@cell content=
					\ifx&##1&% Only make a footnote if the cell is nonempty
						##1
					\else
						\footnote{##1}
					\fi}
			},
		]{platforms.csv}
		\caption{Tools used for graph processing}
		\label{tab:frameworks}
	\end{table}
\end{savenotes}

\section{Conclusion} % Begin with the end in mind...
We have presented an updated survey of parallel graph processing frameworks supplementary to \cite{Doekemeijer:2015:GPFSurvey}. From this, we have selected a representative subset of frameworks on which performance is analyzed and have stored these results in a database. To facilitate parallel graph processing, hardware information and performance results are automatically populated (as were all the tables in this paper). These performance results are then used to provide simple recommendations of the optimally-performing framework given a particular algorithm and problem size.
% We have developed a simple model of hardware and its correlation with performance to predict performance on other architectures

\bibliographystyle{acm}
\bibliography{drp}

\end{document}