\documentclass[11pt]{article}
\usepackage[T1]{fontenc}
\usepackage{lmodern}
\usepackage[margin=1in, headheight=15pt]{geometry}
\usepackage{amsmath, amsthm, amsfonts}
\usepackage[hyphens]{url}
\usepackage{hyperref}
\usepackage{pgfplotstable, booktabs}
\pgfplotsset{compat=1.12}

\usepackage{fancyhdr}
\renewcommand{\headrulewidth}{0pt}
\pagestyle{fancy}
\setlength{\headheight}{15pt} % Removes warnings. Set to 30pt if 2 lines on the header.
\lhead{}
\rhead{}
\usepackage{totcount}
\regtotcounter{page}
\cfoot{\ifnum\totvalue{page} > 1 \thepage \else\fi}

\usepackage{enumitem}

\begin{document}
\title{The State of Graph Processing \\ \large APIs, Libraries, Benchmarks, and Programming Languages \vspace{-1em}}
\author{Samuel Pollard}
\date{\today}
\maketitle

\section{APIs and Libraries}
\begin{itemize}
	\item \textbf{Pregel.} API; developed by Google; distributed; vertex centric, Bulk-Synchronous Parallel model; inspired many other platforms such as Giraph and GPS; \href{http://dl.acm.org/citation.cfm?id=1807184}{original paper}; 2010.

	\item \textbf{GPS (Graph Processing System).} API; developed by Stanford; distributed; vertex centric, Bulk-Synchronous Parallel model; open source; similar to Pregel but with dynamic graph repartitioning and other enhancements. \href{http://ilpubs.stanford.edu:8090/1039/7/gps_ssdbm.pdf}{original paper}; \href{http://infolab.stanford.edu/gps/}{website}; First appeared 2013; apprears inactive.

	\item \textbf{GraphX} Library.
	
	% OpenCL: http://www.lidi.info.unlp.edu.ar/WorldComp2011-Mirror/PDP3248.pdf - Appears inactive since 2011.
\end{itemize}

\section{Programming Languages}
\begin{itemize}
	\item \textbf{GP (Graph Programs).} Nondeterministic; serial; original paper: \cite{Plump:2009:GPL}; website: \url{https://www.cs.york.ac.uk/plasma/wiki/index.php?title=GP_(Graph_Programs)}; appears to be more theoretical and used for program verification; still active.

	\item \textbf{Gremlin.} functional, data flow; distributed; a way to interact with graph databases; \cite{Rodriguez:2015:Gremlin}; website: \url{http://tinkerpop.apache.org/gremlin.html}.
\end{itemize}

\section{Benchmarks}
\begin{itemize}
	\item \textbf{Graphalytics.} CPU and GPU; supports GraphMat, PowerGraph, GraphBIG, Giraph, GraphX, Neo4j, and MapReduce; \cite{Capota:2015:Graphalytics}; still active; so far I can only get PowerGraph and GraphBIG running; website: \url{http://graphalytics.ewi.tudelft.nl}.

	\item \textbf{GAP (Graph Algorithm Platform).} CPU; shared Memory (OpenMP); \url{http://gap.cs.berkeley.edu/benchmark.html}; last active October 2016 on Github. \cite{Beamer:2015:GAPBench}

	\item \textbf{GraphBIG.} CPU and GPU; shared Memory and CUDA; last active February 2016.

	\item \textbf{Lonestar.} CPU and GPU; shared memory and CUDA; part of Galois. First appeared 2011; last update appears to be in 2015 though I am in recent (December 2016) contact with someone working on the project.
\end{itemize}

\section{Dynamic Graphs}
\begin{itemize}
	\item \textbf{STINGER.} Data structure and library; Georgia Institute of Technology and various national laboratories; streaming model; parallel or serial, distributed or shared; \href{http://cass-mt.pnnl.gov/docs/pubs/pnnlgeorgiatechsandiastinger-u.pdf}{original paper}; \href{http://www.stingergraph.com/}{website}; First appeared 2009; still active on Github.

	\item \textbf{GraphJet.} Java library; parallel but single-machine; original purpose was for real time recommendations from Twitter; \url{https://github.com/twitter/GraphJet}; \cite{Sharma:2016:GraphJet}.

	\item \textbf{GraphStream.} Java library; I'm not sure but it appears to be serial; \url{http://graphstream-project.org/}; \cite{Dutot:2007:GraphStream} first active 2007; appears still active.

	\item \textbf{Kineograph.} API; parallel and distributed; \cite{Cheng:2012:Kineograph}; 2012; appears to be inactive but influenced GraphX, GraphChi, and PowerGraph.

	\item \textbf{DynoGraph.} Performance Analysis; \url{https://github.com/sirpoovey/DynoGraph} \cite{Hein:2016:DynoGraph}; connected components, PageRank, BFS; right now just a repo and a poster at SC'16.
\end{itemize}

\section{Visualization}
There are many graph visualization tools out there, most notably Graphviz (and its associated DOT file format). Likewise, databases such as Neo4j have their own visualization tools. This is concerned only with visualizations which are scalable to a large number of vertices and edges.

\begin{itemize}
	\item \textbf{Gephi.} \href{https://gephi.org/}{website}
\end{itemize}


% Other stuff: Giraph++: http://dl.acm.org/citation.cfm?id=2732238
% GraphBuilder: http://dl.acm.org/citation.cfm?id=2484429 and https://01.org/graphbuilder/ - Active from 2013--2014
% Chronos: http://dl.acm.org/citation.cfm?id=2592799
% X-Stream: http://dl.acm.org/citation.cfm?id=2522740

\bibliographystyle{plain}
\bibliography{../report/drp}

\end{document}
